\section*{Analisi del Dominio}
\phantomsection
\addcontentsline{toc}{section}{Analisi del Dominio}

\subsection*{Modello del Dominio}

\subsubsection*{Struttura Account}
\phantomsection
\addcontentsline{toc}{subsection}{Modello del Dominio: Struttura Account}
\vspace{0.5cm}
Questo è il diagramma delle classi del dominio relativo alla struttura degli Account ed alle relazione con gli Attori. \\
Sia Utente che Amministratore estendono Account, in quanto hanno in comune le credenziali di accesso \verb|username| e \verb|password|, omessa per ragioni di sicurezza.\\
Inoltre Amministratore Gruppo estende Utente con cardinalità 0..n - 1, in quanto un solo Utente può essere amministratore di più gruppi.

\vspace{0.5cm}
\begin{adjustwidth}{-2cm}{0cm}
\includegraphics[scale=1]{dominio/Dominio-Struttura Account.drawio.pdf}
\end{adjustwidth}

%------------------------
\pagebreak
\subsubsection*{Area Personale}
\phantomsection
\addcontentsline{toc}{subsection}{Modello del Dominio: Area Personale}
\vspace{0.5cm}
La sezione Area Personale ha come attore l'Utente, il quale usa GestioneFile, GestioneViste, GestioneAccount. Il numero di Viste è limitato a 5, mentre non c'è un limite nel numero di File che è legato solo allo spazio disponibile residuo dell'Utente.
\vspace{0.5cm}

\begin{adjustwidth}{-2cm}{0cm}
\includegraphics[scale=0.9]{dominio/Dominio-Area Personale.drawio.pdf}
\end{adjustwidth}


%-----------------
\pagebreak
\subsubsection*{Area Gruppi}
\phantomsection
\addcontentsline{toc}{subsection}{Modello del Dominio: Area Gruppi}
\vspace{0.5cm}
Nell' Area Gruppi l'Utente ha a disposizione i metodi per gestire all'interno dei gruppi le proprie risorse e viste. L'Utente dispone anche dei metodi per aggiornare la sua configurazione dell'account tramite GestioneAccount, il quale espone anche lo spazio rimanente.
\vspace{0.5cm}

\begin{adjustwidth}{-2cm}{0cm}
\includegraphics[scale=1]{dominio/Dominio-Area Gruppi.drawio.pdf}
\end{adjustwidth}

%-----------------
\pagebreak
\subsubsection*{Amministrazione}
\phantomsection
\addcontentsline{toc}{subsection}{Modello del Dominio: Amministrazione}
\vspace{0.5cm}

Nella sezione Amministratore, si dispongono i metodi per gestire le richieste.\linebreak \textbf{gestisciRichiestaUtente()} e \textbf{gestisciRichiestaGruppo()} servono per accettare o rifiutare le richieste "in attesa". L'Amministratore può anche eliminare Utenti e Gruppi.Le operazioni sono tracciate registrando nei Logs ogni richiesta.

\vspace{0.5cm}
\begin{adjustwidth}{-2.5cm}{0cm}
\includegraphics[scale=0.9]{dominio/Dominio-Amministrazione.drawio.pdf}
\end{adjustwidth}
\pagebreak

%------------------
\phantomsection
\addcontentsline{toc}{section}{Architettura Logica: Struttura}
\includepdf[pages={1}]{classi/Package-Package.drawio.pdf}
%-------------------

\subsection*{Diagrammi delle Classi}

\phantomsection
\subsubsection*{Diagramma delle Classi: Autenticazione}
\addcontentsline{toc}{subsection}{Diagramma delle Classi: Autenticazione}
\vspace{1cm}
Il metodo \verb|registraUtente| inserisce la richiesta di registrazione nella \\\verb|listaRichiesteRegistrazione| della classe \verb|GestioneAmministrazioneController|.\\
L' operazione portata a termine quando l'Amministratore accetta la richiesta.
\vspace{2cm}
\begin{adjustwidth}{-2.5cm}{0cm}
\includegraphics[scale=0.7]{classi/Package-Classi-Autenticazione.drawio.pdf}
\end{adjustwidth}



\phantomsection
\subsubsection*{Diagramma delle Classi: Amministrazione}
\addcontentsline{toc}{subsection}{Diagramma delle Classi: Amministrazione}
\vspace{0.5cm}

\begin{adjustwidth}{-2.5cm}{0cm}
\includegraphics[scale=0.8]{classi/Package-Classi-Amministrazione.drawio.pdf}
\end{adjustwidth}
\vspace{0.5cm}




\phantomsection
\subsubsection*{Diagramma delle Classi: Utente}
\addcontentsline{toc}{subsection}{Diagramma delle Classi: Utente}
\vspace{0.5cm}


\vspace{0.5cm}
\begin{adjustwidth}{-3.5cm}{0cm}
\includegraphics[scale=0.7]{classi/Package-Classi-Utente.drawio.pdf}
\end{adjustwidth}




\phantomsection
\subsubsection*{Diagramma delle Classi: Gruppi}
\addcontentsline{toc}{subsection}{Diagramma delle Classi: Gruppi}
\vspace{0.5cm}


\vspace{0.5cm}
\begin{adjustwidth}{-2.5cm}{0cm}
\includegraphics[scale=0.75]{classi/Package-Classi-Gruppi.drawio.pdf}
\end{adjustwidth}



%------------------------------------------


\phantomsection
\section*{Architettura Logica: Interazione}
\addcontentsline{toc}{section}{Architettura Logica: Interazione}
\vspace{0.5cm}


\phantomsection
\subsection*{Diagramma di sequenza : Registrazione}
\addcontentsline{toc}{subsection}{Diagramma di sequenza : Registrazione}
\vspace{0.5cm}

\begin{adjustwidth}{-0.5cm}{0cm}
\includegraphics[scale=0.9]{interazione/Package-Interazione-Registrazione.drawio.pdf}
\end{adjustwidth}

\phantomsection
\subsection*{Diagramma di sequenza : Autenticazione}
\addcontentsline{toc}{subsection}{Diagramma di sequenza : Autenticazione}
\vspace{0.5cm}

\begin{adjustwidth}{-0.5cm}{0cm}
\includegraphics[scale=0.9]{interazione/Package-Interazione-Autenticazione.drawio.pdf}
\end{adjustwidth}


\phantomsection
\subsection*{Diagramma di sequenza : Gestione Amministratore}
\addcontentsline{toc}{subsection}{Diagramma di sequenza : Gestione Amministratore}
\vspace{0.5cm}

\vspace{0.5cm}
\begin{adjustwidth}{-1cm}{0cm}
\includegraphics[scale=0.8]{interazione/Package-Interazione-GestioneAmministratore.drawio.pdf}
\end{adjustwidth}

\phantomsection
\subsubsection*{Diagramma di sequenza : Area Personale\-Viste}
\addcontentsline{toc}{subsection}{Diagramma di sequenza : Area Personale\-Viste}
\vspace{0.5cm}

\begin{adjustwidth}{-2.5cm}{0cm}
\includegraphics[scale=0.8]{interazione/Package-Interazione-AreaPersonale.drawio.pdf}
\end{adjustwidth}


\phantomsection
\subsubsection*{Diagramma di sequenza : Area Gruppi}
\addcontentsline{toc}{subsection}{Diagramma di sequenza : Area Gruppi}
\vspace{0.5cm}

\begin{adjustwidth}{-2.5cm}{0cm}
\includegraphics[scale=0.8]{interazione/Package-Interazione-AreaGruppi.drawio.pdf}
\end{adjustwidth}
