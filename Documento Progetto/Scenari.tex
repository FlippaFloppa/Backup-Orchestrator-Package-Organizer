\phantomsection
\section*{Scenari}
\addcontentsline{toc}{section}{Scenari}

\phantomsection
\addcontentsline{toc}{subsection}{Registrazione}
\rowcolors{2}{white!0!}{white!0!}
\begin{adjustwidth}{0cm}{0cm}

\resizebox{1\textwidth}{!}{
\begin{tabular}{ |l|l|  }
\hline
\rowcolor{red!25}\large\textbf{Titolo} & \large{Registrazione}\\
\hline

\textbf{Descrizione} & L'Utente si registra\\
\hline
\textbf{Attori} & Utente\\
\hline
\textbf{Relazioni} & \\
\hline
\textbf{Precondizioni} & \\
\hline
\textbf{Postcondizioni} & \begin{tabular}{@{}l@{}} L'Utente viene registrato nel sistema e gli viene allocato lo spazio di \\ archiviazione predefinito. Necessaria approvazione da parte dell'Amministratore\end{tabular} \\
\hline
\textbf{Scenario principale} &
\begin{tabular}{@{}l@{}}
1) Si presenta all' Utente un'interfaccia per la registrazione dell' account\\  
2) L'Utente inserisce le credenziali per la creazione dell' account\\
3) Il sistema controlla che l'Utente non sia già registrato\\
4) La richiesta viene memorizzata nel sistema.
\end{tabular} \\
\hline
\textbf{Scenari alternativi} &
\begin{tabular}{@{}l@{}}
Scenario A: L'username scelto è già registrato: \\
\quad 1) L'Utente viene notificato dal servizio dell' errore\\
\quad 2) L'Utente cambia username e riprova la registrazione\\
\end{tabular} \\
\hline
\textbf{Requisiti non funzionali} & Interfaccia Utente intuitiva\\
\hline
\textbf{Punti aperti} & \\
\hline

\end{tabular}
}
\end{adjustwidth}
\vspace{0.2cm}


\phantomsection
\addcontentsline{toc}{subsection}{Autenticazione}
\rowcolors{2}{white!0!}{white!0!}
\begin{adjustwidth}{0cm}{0cm}

\resizebox{1\textwidth}{!}{
\begin{tabular}{ |l|l|  }
\hline
\rowcolor{red!25}\large\textbf{Titolo} & \large{Autenticazione}\\
\hline

\textbf{Descrizione} & L'Utente effettua il login nel sistema\\
\hline
\textbf{Attori} & Utente, Amministratore, Amministratore Gruppo\\
\hline
\textbf{Relazioni} &\begin{tabular}{@{}l@{}} AreaPersonale, AreaGruppi, GestioneUtenti,\\ GestioneRichieste, GestioneGruppi, GestioneGruppo\end{tabular} \\ 
\hline
\textbf{Precondizioni} & \begin{tabular}{@{}l@{}} L'Utente / Amministratore deve essere registrato nel sistema\end{tabular} \\
\hline
\textbf{Postcondizioni} & \\
\hline
\textbf{Scenario principale} &
\begin{tabular}{@{}l@{}}
1) Si presenta all'Utente / Amministratore un'interfaccia \\per l'autenticazione dell'account\\
2) L'Utente / Amministratore fornisce le credenziali \\
3) Il sistema effettua le opportune verifiche e, in caso di \\ approvazione, presenta la maschera dell'area pesonale
\end{tabular} \\
\hline
\textbf{Scenari alternativi} & 
\begin{tabular}{@{}l@{}}
Scenario A: Credenziali errate\\
\quad - Viene presentata nuovamente la maschera di Login
\end{tabular}\\
\hline
\textbf{Requisiti non funzionali} & Interfaccia utente intuitiva \\
\hline
\textbf{Punti aperti} & \\
\hline

\end{tabular}
}
\end{adjustwidth}
\vspace{0.2cm}


\phantomsection
\addcontentsline{toc}{subsection}{GestioneUtenti}
\rowcolors{2}{white!0!}{white!0!}
\begin{adjustwidth}{0cm}{0cm}

\resizebox{1\textwidth}{!}{
\begin{tabular}{ |l|l|  }
\hline
\rowcolor{red!25}\large\textbf{Titolo} & \large{GestioneUtenti}\\
\hline

\textbf{Descrizione} & Sezione per gestire gli utenti registrati\\
\hline
\textbf{Attori} & Amministratore\\
\hline
\textbf{Relazioni} & Autenticazione\\
\hline
\textbf{Precondizioni} &\\
\hline
\textbf{Postcondizioni} & \\
\hline
\textbf{Scenario principale} & \begin{tabular}{@{}l@{}} 
1) Autenticazione\\
2) L'Amministratore si sposta sulla sezione dedicata alla moderazione degli utenti\\
3) Il sistema mostra la lista degli utenti registrati\\
4) L'Amministratore può decidere di effettuare le seguenti operazioni:\\
\quad - Eliminare un Utente\\
\end{tabular}\\
\hline
\textbf{Scenari alternativi} & \\
\hline
\textbf{Requisiti non funzionali} & Pannello di amministrazione intuitivo\\
\hline
\textbf{Punti aperti} & \begin{tabular}{@{}l@{}} 
Altre operazioni eseguibili dall'Amministratore:\\
\quad - Modificare la quantità di memoria disponibile per l'Utente\\
\end{tabular}\\
\hline

\end{tabular}
}
\end{adjustwidth}
\vspace{0.5 cm}






\phantomsection
\addcontentsline{toc}{subsection}{GestioneGruppi}
\rowcolors{2}{white!0!}{white!0!}
\begin{adjustwidth}{0cm}{0cm}
\resizebox{1\textwidth}{!}{
\begin{tabular}{ |l|l|  }
\hline
\rowcolor{red!25}\large\textbf{Titolo} & \large{GestioneGruppi}\\
\hline
\textbf{Descrizione} & Sezione per gestire i gruppi degli utenti registrati\\
\hline
\textbf{Attori} & Amministratore\\
\hline
\textbf{Relazioni} & Autenticazione\\
\hline
\textbf{Precondizioni} &\\
\hline
\textbf{Postcondizioni} & \\
\hline
\textbf{Scenario principale} & \begin{tabular}{@{}l@{}} 
1) Autenticazione\\
2) L'Amministratore si sposta sulla sezione dedicata alla moderazione dei gruppi\\
3) Il sistema mostra la lista dei gruppi registrati\\
4) L'Amministratore può decidere di effettuare le seguenti operazioni:\\
\quad - Eliminare un gruppo\\
\end{tabular}\\
\hline
\textbf{Scenari alternativi} & \\
\hline
\textbf{Requisiti non funzionali} & Pannello di amministrazione intuitivo\\
\hline
\textbf{Punti aperti} & \begin{tabular}{@{}l@{}} 
- L'Amministratore può modificare la quantità di memoria disponibile per\\ il gruppo.\\
- L'Amministratore può garantire permessi sui file agli Utenti del gruppo.\\
\end{tabular}\\
\hline

\end{tabular}
}
\end{adjustwidth}
\vspace{0.2cm}





\phantomsection
\addcontentsline{toc}{subsection}{GestioneRichieste}
\rowcolors{2}{white!0!}{white!0!}
\begin{adjustwidth}{0cm}{0cm}

\resizebox{1\textwidth}{!}{
\begin{tabular}{ |l|l|  }
\hline
\rowcolor{red!25}\large\textbf{Titolo} & \large{GestioneRichieste}\\
\hline

\textbf{Descrizione} & Gestisce le richieste degli utenti e dei gruppi\\
\hline
\textbf{Attori} & Amministratore\\
\hline
\textbf{Relazioni} & Autenticazione, CreazioneGruppo, RegistrazioneUtente\\
\hline
\textbf{Precondizioni} &\\
\hline
\textbf{Postcondizioni} &\begin{tabular}{@{}l@{}} \end{tabular} \\
\hline
\textbf{Scenario principale} & \begin{tabular}{@{}l@{}} 
1) Autenticazione\\
2) L'Amministratore si sposta sulla sezione dedicata alla moderazione\\
3) L'Amministratore può spostarsi sulle sezioni dedicate alla gestione\\
\quad delle richieste di registrazione degli utenti o della creazione di gruppi\\
\end{tabular}\\
\hline
\textbf{Scenari alternativi} & \\
\hline
\textbf{Requisiti non funzionali} & Pannello di amministrazione intuitivo\\
\hline
\textbf{Punti aperti} & \\
\hline

\end{tabular}
}
\end{adjustwidth}
\vspace{0.2cm}



\phantomsection
\addcontentsline{toc}{subsection}{RegistrazioneUtente}
\rowcolors{2}{white!0!}{white!0!}
\begin{adjustwidth}{0cm}{0cm}

\resizebox{1\textwidth}{!}{
\begin{tabular}{ |l|l|  }
\hline
\rowcolor{red!25}\large\textbf{Titolo} & \large{RegistrazioneUtente}\\
\hline

\textbf{Descrizione} & Gestisce le di registrazione degli Utenti\\
\hline
\textbf{Attori} & Amministratore\\
\hline
\textbf{Relazioni} & GestioneRichieste\\
\hline
\textbf{Precondizioni} &\\
\hline
\textbf{Postcondizioni} &\begin{tabular}{@{}l@{}} \end{tabular} \\
\hline
\textbf{Scenario principale} & \begin{tabular}{@{}l@{}} 
1) L'Amministratore si è spostato sulla sezione per gestire le richieste di registrazione degli Utenti\\
2) L'Amministratore può decidere di effettuare le seguenti operazioni:\\
\quad - Accettare le richieste\\
\quad - Rifiutare le richieste\\
\end{tabular}\\
\hline
\textbf{Scenari alternativi} & \\
\hline
\textbf{Requisiti non funzionali} & Pannello di amministrazione intuitivo\\
\hline
\textbf{Punti aperti} & \\
\hline

\end{tabular}
}
\end{adjustwidth}
\vspace{0.2cm}


\phantomsection
\addcontentsline{toc}{subsection}{CreazioneGruppo}
\rowcolors{2}{white!0!}{white!0!}
\begin{adjustwidth}{0cm}{0cm}

\resizebox{1\textwidth}{!}{
\begin{tabular}{ |l|l|  }
\hline
\rowcolor{red!25}\large\textbf{Titolo} & \large{CreazioneGruppo}\\
\hline

\textbf{Descrizione} & Gestisce le di creazione dei Gruppi\\
\hline
\textbf{Attori} & Amministratore\\
\hline
\textbf{Relazioni} & GestioneRichieste\\
\hline
\textbf{Precondizioni} &\\
\hline
\textbf{Postcondizioni} &\begin{tabular}{@{}l@{}} \end{tabular} \\
\hline
\textbf{Scenario principale} & \begin{tabular}{@{}l@{}} 
1) L'Amministratore si è spostato sulla sezione per gestire le richieste di creazione dei Gruppi\\
2) L'Amministratore può decidere di effettuare le seguenti operazioni:\\
\quad - Accettare le richieste\\
\quad - Rifiutare le richieste\\
\end{tabular}\\
\hline
\textbf{Scenari alternativi} & \\
\hline
\textbf{Requisiti non funzionali} & Pannello di amministrazione intuitivo\\
\hline
\textbf{Punti aperti} & \\
\hline

\end{tabular}
}
\end{adjustwidth}
\vspace{0.2cm}




\phantomsection
\addcontentsline{toc}{subsection}{AreaPersonale}
\rowcolors{2}{white!0!}{white!0!}
\begin{adjustwidth}{0cm}{0cm}

\resizebox{1\textwidth}{!}{
\begin{tabular}{ |l|l|  }
\hline
\rowcolor{red!25}\large\textbf{Titolo} & \large{AreaPersonale}\\
\hline

\textbf{Descrizione} & Area Personale dell'Utente\\
\hline
\textbf{Attori} & Utente\\
\hline
\textbf{Relazioni} & Autenticazione, GestioneAccount, GestioneViste, GestioneFile\\
\hline
\textbf{Precondizioni} &\\
\hline
\textbf{Postcondizioni} & \\
\hline
\textbf{Scenario principale} & 
\begin{tabular}{@{}l@{}} 
1) Autenticazione\\
2) All'Utente viene presentata la pagina dell'area personale.\\
3) L'Utente potrà spostarsi su tre diverse sezioni:\\
\quad - Gestione Account\\
\quad - Gestione File personali\\
\quad - Gestione Viste\\
\end{tabular}\\
\hline
\textbf{Scenari alternativi} & \\
\hline
\textbf{Requisiti non funzionali} & Interfaccia utente intuitiva\\
\hline
\textbf{Punti aperti} & \\
\hline

\end{tabular}
}
\end{adjustwidth}
\vspace{0.2cm}





\phantomsection
\addcontentsline{toc}{subsection}{GestioneAccount}
\rowcolors{2}{white!0!}{white!0!}
\begin{adjustwidth}{0cm}{0cm}

\resizebox{1\textwidth}{!}{
\begin{tabular}{ |l|l|  }
\hline
\rowcolor{red!25}\large\textbf{Titolo} & \large{GestioneAccount}\\
\hline

\textbf{Descrizione} & Gestione dell'account personale dell'Utente\\
\hline
\textbf{Attori} & Utente\\
\hline
\textbf{Relazioni} & AreaPersonale\\
\hline
\textbf{Precondizioni} &\\
\hline
\textbf{Postcondizioni} & \\
\hline
\textbf{Scenario principale} & \begin{tabular}{@{}l@{}}
1) L'Utente si sposta sulla sezione dedicata alla gestione dell'account personale\\
2) L'Utente ha a disposizione le seguenti operazioni:\\
\quad - Modifica nickname\\
\quad - Modifica password\\
\quad - Eliminazione dell'account\\
\end{tabular}\\
\hline
\textbf{Scenari alternativi} & \\
\hline
\textbf{Requisiti non funzionali} & \\
\hline
\textbf{Punti aperti} & L'Utente può modificare l'username\\
\hline

\end{tabular}
}
\end{adjustwidth}
\vspace{0.2cm}



\phantomsection
\addcontentsline{toc}{subsection}{GestioneViste}
\rowcolors{2}{white!0!}{white!0!}
\begin{adjustwidth}{0cm}{0cm}

\resizebox{1\textwidth}{!}{
\begin{tabular}{ |l|l|  }
\hline
\rowcolor{red!25}\large\textbf{Titolo} & \large{GestioneViste}\\
\hline

\textbf{Descrizione} & Sezione in cui l'Utente modifica le viste\\
\hline
\textbf{Attori} & Utente\\
\hline
\textbf{Relazioni} & AreaPersonale\\
\hline
\textbf{Precondizioni} &\\
\hline
\textbf{Postcondizioni} &\\
\hline
\textbf{Scenario principale} &
\begin{tabular}{@{}l@{}} 
1) L'Utente si sposta sulla sezione dedicata alle viste\\
2) Il sistema presenta all'Utente la lista di viste salvate ed un editor di viste\\
3) L'Utente può:\\
\quad - Creare una nuova vista\\
\quad - Modificare una vista\\
\quad - Eliminare una vista\\
\end{tabular}\\
\hline
\textbf{Scenari alternativi} &
\begin{tabular}{@{}l@{}} 
Scenario A: Raggiunto il limite massimo di viste\\
\quad - È possibile modificare ed eliminare le viste, ma non crearne di nuove\\
\end{tabular}\\
\hline
\textbf{Requisiti non funzionali} & \\
\hline
\textbf{Punti aperti} & Editor grafico delle viste\\
\hline

\end{tabular}
}
\end{adjustwidth}
\vspace{0.2cm}





\phantomsection
\addcontentsline{toc}{subsection}{GestioneFile}
\rowcolors{2}{white!0!}{white!0!}
\begin{adjustwidth}{0cm}{0cm}

\resizebox{1\textwidth}{!}{
\begin{tabular}{ |l|l|  }
\hline
\rowcolor{red!25}\large\textbf{Titolo} & \large{GestioneFile}\\
\hline

\textbf{Descrizione} & Sezione in cui l'Utente allo spazio di archiviazione personale\\
\hline
\textbf{Attori} & Utente\\
\hline
\textbf{Relazioni} & AreaPersonale\\
\hline
\textbf{Precondizioni} & \\
\hline
\textbf{Postcondizioni} &\\
\hline
\textbf{Scenario principale} &
\begin{tabular}{@{}l@{}} 
1) L'Utente si sposta sulla sezione dedicata allo storage\\
2) Il sistema presenta all'Utente il file manager dei file personali archiviati \\nel sistema\\
3) L'Utente può:\\
\quad - Visualizzare i file/cartelle che ha caricato\\
\quad - Caricare uno o più file/cartelle\\
\quad - Scaricare un file/cartella\\
\quad - Eliminare file/cartelle\\
\quad - Modificare la visualizzazione dei file selezionando la vista opportuna\\
\end{tabular}\\
\hline
\textbf{Scenari alternativi} &
\begin{tabular}{@{}l@{}} 
Scenario A: Raggiunto il limite massimo di memoria di archiviazione\\
\quad - Non è possibile caricare nuovi file/cartelle\\
\end{tabular}\\
\hline
\textbf{Requisiti non funzionali} & 
\begin{tabular}{@{}l@{}} 
- Velocità nel caricare/scaricare file\\
- Organizzazione del file manager ampiamente personalizzabile\\
- Possibilità di cambiare le view in modo rapido\\
- Utilizzo efficace del servizio anche in assenza di una connessione veloce\\
\end{tabular}\\
\hline
\textbf{Punti aperti} & Anteprima file\\
\hline

\end{tabular}
}
\end{adjustwidth}
\vspace{0.2cm}



\phantomsection
\addcontentsline{toc}{subsection}{AreaGruppi}
\rowcolors{2}{white!0!}{white!0!}
\begin{adjustwidth}{0cm}{0cm}

\resizebox{1\textwidth}{!}{
\begin{tabular}{ |l|l|  }
\hline
\rowcolor{red!25}\large\textbf{Titolo} & \large{AreaGruppi}\\
\hline

\textbf{Descrizione} & Sezione dedicata ai gruppi di più utenti\\
\hline
\textbf{Attori} & Utente\\
\hline
\textbf{Relazioni} & Autenticazione, Gruppo\\
\hline
\textbf{Precondizioni} & \\
\hline
\textbf{Postcondizioni} & \\
\hline
\textbf{Scenario principale} &
\begin{tabular}{@{}l@{}}
1) Autenticazione\\
2) L'Utente si sposta sulla sezione dedicata ai gruppi\\
3) Viene mostrata la lista dei gruppi registrati nel sistema\\
4) L'Utente può:\\
\quad - Iscriversi ad un gruppo, necessaria password\\
\quad - Creare un gruppo, la richiesta deve essere approvata da un Amministratore\\
\quad - Esplorare un gruppo\\
\end{tabular}\\
\hline
\textbf{Scenari alternativi} & \\
\hline
\textbf{Requisiti non funzionali} & \\
\hline
\textbf{Punti aperti} & \\
\hline

\end{tabular}
}
\end{adjustwidth}
\vspace{0.2cm}



\phantomsection
\addcontentsline{toc}{subsection}{Gruppo}
\rowcolors{2}{white!0!}{white!0!}
\begin{adjustwidth}{0cm}{0cm}

\resizebox{1\textwidth}{!}{
\begin{tabular}{ |l|l|  }
\hline
\rowcolor{red!25}\large\textbf{Titolo} & \large{Gruppo}\\
\hline

\textbf{Descrizione} & Pagina del gruppo\\
\hline
\textbf{Attori} & Utente\\
\hline
\textbf{Relazioni} & AreaGruppi\\
\hline
\textbf{Precondizioni} & \\
\hline
\textbf{Postcondizioni} & \\
\hline
\textbf{Scenario principale} &
\begin{tabular}{@{}l@{}}
1) L'Utente entra nella sezione dedicata ad un gruppo\\
2) L'Utente può:\\
\quad - Visualizzare la lista degli utenti iscritti al gruppo\\
\quad - Eseguire operazioni sui file del gruppo\\
\quad - Uscire dal gruppo\\
\end{tabular}\\
\hline
\textbf{Scenari alternativi} & \\
\hline
\textbf{Requisiti non funzionali} & \\
\hline
\textbf{Punti aperti} & \\
\hline

\end{tabular}
}
\end{adjustwidth}
\vspace{0.2cm}




\phantomsection
\addcontentsline{toc}{subsection}{GestioneGruppo}
\rowcolors{2}{white!0!}{white!0!}
\begin{adjustwidth}{0cm}{0cm}

\resizebox{1\textwidth}{!}{
\begin{tabular}{ |l|l|  }
\hline
\rowcolor{red!25}\large\textbf{Titolo} & \large{GestioneGruppo}\\
\hline

\textbf{Descrizione} & Pagina di amministrazione del gruppo\\
\hline
\textbf{Attori} & Amministratore Gruppo\\
\hline
\textbf{Relazioni} & Autenticazione\\
\hline
\textbf{Precondizioni} & \\
\hline
\textbf{Postcondizioni} & \\
\hline
\textbf{Scenario principale} &
\begin{tabular}{@{}l@{}}
1) Autenticazione\\
2) L'Amministratore del gruppo accede alla sezione del gruppo\\ di cui è Amministratore\\
2) L'Amministratore del gruppo può:\\
\quad - Visualizzare la lista degli utenti iscritti al gruppo\\
\quad - Aggiungere un Utente al gruppo\\
\quad - Rimuovere un Utente dal gruppo\\
\end{tabular}\\
\hline
\textbf{Scenari alternativi} & \\
\hline
\textbf{Requisiti non funzionali} & \\
\hline
\textbf{Punti aperti} & \\
\hline

\end{tabular}
}
\end{adjustwidth}
\vspace{0.2cm}
